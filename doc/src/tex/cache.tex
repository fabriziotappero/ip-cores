\chapter{Cache/Memory Controller Module}
\label{cache}

    The project includes a cache+memory controller module from revision 114.\\
        
    Both the I- and the D-Cache are implemented. But the parametrization 
    generics are still mostly unused, with many values hardcoded. And SDRAM is 
    not supported yet. Besides, there are some loose ends in the implementation 
    still to be solved, exlained in section ~\ref{cache_problems}.\\
    
    
\section{Cache Initialization and Control}
\label{cache_init_and_control}

    The cache module comes up from reset in an indeterminate, unuseable state.
    It needs to be initialized before being enabled.\\
    Initialization means mostly marking all D- and I-cache lines as invalid. 
    The old R3000 had its own means to achieve this, but this core implements an
    alternative, simplified scheme.\\

    The standard R3000 cache control flags in the SR are not used, either. Instead,
    two flags from the SR have been repurposed for cache control.\\

\subsection{Cache control flags}
\label{cache_control_flags}
    
    Bits 17 and 16 of the SR are NOT used for their standard R3000 purpose.
    Instead they are used as explained below:
    
    \begin{itemize}
    \item Bit 17: Cache enable              [reset value = 0]
    \item Bit 16: I- and D-Cache line invalidate   [reset value = 0]
    \end{itemize}

    You always use both these flags together to set the cache operating mode:
    
    \begin{itemize}
    \item Bits 17:16='00'\\
        Cache is enabled and working.
    \item Bits 17:16='01'\\
        Cache is in D- and I-cache line invalidation mode.\\
        Writing word X.X.X.N to ANY address will 
        invalidate I-Cache line N (N is an 8-bit word and X is an 8-bit 
        don't care). Besides, the actual write will be performed too; be 
        careful...\\
    
        Reading from any address will cause the 
        corresponding D-cache line to be invalidated; the read will not be
        actually performed and the read value is undefined.\\
    \item Bits 17:16='10'\\
        Cache is disabled; no lines are being invalidated.\\
    \item Bits 17:16='11'\\
        Cache behavior is UNDETERMINED -- i.e. guaranteed crash.\\
    \end{itemize}
    
    Now, after reset the cache memory comes up in an undetermined state but 
    it comes up disabled too. Before enabling it, you need to invalidate all
    cache lines in software (see routine cache\_init in the memtest sample).\\

\section{Cache Tags and Cache Address Mirroring}
\label{cache_tags}

    In order to save space in the I-Cache tag table, the tags are shorter than 
    they should -- 14 bits instead of the 20 bits we would need to cover the
    entire 32-bit address:

\needspace{10\baselineskip}
\begin{verbatim}

                _________ <-- These address bits are NOT in the tag
               /         \
    31 ..   27| 26 .. 21  |20 ..          12|11  ..        4|3:2|
    +---------+-----------+-----------------+---------------+---+---+
    | 5       |           | 9               | 8             | 2 |   |
    +---------+-----------+-----------------+---------------+---+---+
    ^                     ^                 ^               ^- LINE_INDEX_SIZE
    5 bits                9 bits            LINE_NUMBER_SIZE

    \end{verbatim}\\
    
    Since bits 26 downto 21 are not included in the tag, there will be a 
    'mirror' effect in the cache. We have effectively split the memory space 
    into 32 separate blocks of 1MB which is obviously not enough but will do
    for the initial versions of the core.
    
    In subsequent versions of the cache, the tag size needs to be enlarged AND 
    some of the top bits might be omitted when they're not needed to implement 
    the default MIPS memory map (namely bit 30 which is always '0').

    
\section{Memory Controller}
\label{memory_controller}

    The cache functionality and the memory controller functionality are so 
    closely related that I found it convenient to bundle both in the same 
    module: I have experimented with separate modules and was unable to come up
    with the same performance with my synthesis tools of choice.
    So, while it would be desirable to split the cache from the memory controller
    at some later version, for the time being they are a single module.\\

    The memory controller interfaces the cache to external memory, be it off-core 
    or off-chip.
    The memory controller implements the refill and writethrough state machines,
    that will necessarily be different for different kinds of memory.\\


\subsection{Memory Map Definition}
\label{memory_map_definition}    
    
    The MIPS architecture specs define a memory map which determines which areas
    are cached and which is the default address translation \cite[p.~2-8]{r3k_ref_man}.\\
    Neither the memory translation nor the cacheable/uncacheable attribute of
    the standard MIPS architecture have been implemented. In this core, program 
    addresses are always identical to hardware addresses.\\

    When requested to perform a refill or a writethrough, the memory controller 
    needs to know what type of memory it is to be dealing with. The type of
    memory for each memory area is defined in a hardcoded memory map 
    implemented in function 'decode\_address\_mips1', defined in package 
    'mips\_pkg.vhdl'. This function will synthesize into regular, combinational 
    decode logic.\\

    For each address, the memory map logic will supply the following information:

\begin{enumerate}
    \item What kind of memory it is.
    \item How many wait states to use.
    \item Whether it is writeable or not (ignored in current version).
    \item Whether it is cacheable or not (ignored in current version).
\end{enumerate}
        
    In the present implementation the memory map can't be modified at run time.\\
    
    These are the currently supported memory types:
    
\begin{tabular}{ll}
\hline
Identifier & Description \\
\hline
MT\_BRAM            & Synchronous, on-chip FPGA BRAM\\
MT\_IO\_SYNC        & Synchronous, on-chip register (meant for I/O)\\
MT\_SRAM\_16B       & Asynchronous, off-chip static memory, 16-bit wide\\
MT\_SRAM\_8B        & Asynchronous, off-chip static memory, 8-bit wide\\
MT\_DDR\_16B        & Unused yet\\
MT\_UNMAPPED        & Unmapped area\\
\hline
\end{tabular}
   
\subsection{Invalid memory accesses}
\label{invalid_memory}

    Whenever the CPU attempts an invalid memory access, the 'unmapped' output 
    port of the Cache module will be asserted for 1 clock cycle.
    
    The accesses that will raise the 'unmapped' output are these:
    
    \begin{enumerate}
    \item Code fetch from address decoded as MT\_IO\_SYNC.
    \item Data write to memory address decoded as other than RAM or IO.
    \item Any access to an address decoded as MT\_UNMAPPED.
    \end{enumerate}
    
    The 'unmapped' output is ignored by the current version of the parent MCU 
    module -- it is only used to raise a permanent flag that is then connected
    to a LED for debugging purposes, hardly a useful approach in a real project.
    
    In subsequent versions of the MCU module, the 'unmapped' signal will trigger
    a hardware interrupt.
    
    Note again that the memory attributes 'writeable' and 'cacheable' are not
    used in the current version. In subsequent versions 'writeable' will be 
    used and 'cacheable' will be removed.
    
\subsection{Uncacheable memory areas}
\label{uncacheable_memory}

    There are no predefined 'uncacheable' memory areas in the current version of
    the core; all memory addresses are cacheable unless they are defined as
    IO (see mips\_cache.vhdl).\\
    In short, if it's not defined as MT\_UNMAPPED or MT\_IO\_SYNC, it is 
    cacheable.

\section{Cache Refill and Writethrough Chronograms}
\label{cache_state_machine}


The cache state machine deals with cache refills and data writethroughs. It is
this state machine that 'knows' about the type and size of the external 
memories. when DRAM is eventually implemented, or when other widths of SRAM are
supported, this state machine will need to change.

The following subsections will describe the refill and writethrough procedures.
Bear in mind that there is a single state machine that handles it all -- refills
and writethroughs can't be done simultaneously.


\subsubsection{SRAM interface read cycle timing -- 16-bit interface}
\label{sram_read_cycle_16b}

The refill procedure is identical for both D- and I-cache refills. All that 
matters is the type of memory being read. 



\needspace{15\baselineskip}
\begin{verbatim}
==== Chronogram 4.1: 16-bit SRAM refill -- DATA ============================
              __    __    __    __    __    __    _     __    __    __    __
clk         _/  \__/  \__/  \__/  \__/  \__/  \__/ ..._/  \__/  \__/  \__/ 

cache/ps    ?| (1)             | (2)             | ... | (2)             |??

refill_ctr  ?| 0                                 | ... |  3              |??

chip_addr   ?|  210h           |  211h           | ... |  217h           |--

data_rd     -XXXXX  [218h]     XXXXX  [219h]     | ... XXXXX  [217h]     |-- 
             |<- 2-state sequence              ->| 
            _                                                             __
sram_oe_n    \____________________________________ ... __________________/  
             |<- Total: 24 clock cycles to refill a 4-word cache line  ->|  
============================================================================
\end{verbatim}\\

(NOTE: signal names left-clipped to fit page margins)

In the diagram, the data coming into bram\_data\_rd is depicted with some delay.

Signal \emph{cache/ps} is the current state of the cache state machine, and
in this chronogram it takes the following values:

\begin{enumerate}
\item data\_refill\_sram\_0
\item data\_refill\_sram\_1
\end{enumerate}

Each of the two states reads a halfword from SRAM. The two-state sequence is 
repeated four times to refill the four-word cache entry.

Signal \emph{refill\_ctr} counts the word index within the line being refilled,
and runs from 0 to 4.\\


\subsubsection{SRAM interface read cycle timing -- 8-bit interface}
\label{sram_read_cycle_8b}

The refill from an 8-bit static memory is essentially the same as depicted 
above, except we need to read 4 bytes (over the LSB lines of the static memory 
data bus) instead of 2 16-bit halfwords. The operation takes correspondingly
longer to perform and uses an extra address line but is otherwise identical.

TODO: 8-bit refill chronogram to be done.


\subsubsection{16-bit SRAM interface write cycle timing}
\label{sram_write_cycle}

The path of the state machine that deals with SRAM writethroughs is linear so 
a state diagram would not be very interesting. As you can see in the source 
code, all the states are one clock long except for states 
\emph{data\_writethrough\_sram\_0b} and \emph{data\_writethrough\_sram\_1b}, 
which will be as long as the number of wait states plus one.
This is the only writethrough parameter that is influenced by the wait state 
attribute.\\

A general memory write will be 32-bit wide and thus it will take two 16-bit 
memory accesses to complete. Unaligned, halfword or byte wide CPU writes might 
in some cases be optimized to take only a single 16-bit memory access. This 
module does no such optimization yet.
For simplicity, all writethroughs take two 16-bit access cycles, even if one
of them has both we\_n signals deasserted.\\

The following chronogram has been copied from a simulation of the 'hello' 
sample. It's a 32-bit wide write to address 00000430h.
As you can see, the 'chip address' (the address fed to the SRAM chip) is the 
target address divided by 2 (because there are 2 16-bit halfwords to the 32-bit 
word). In this particular case, all the four bytes of the long word are written 
and so both the we\_n signals are asserted for both halfwords.

In this example, the SRAM is being accessed with 1 WS: WE\_N is asserted for
two cycles. 
Note how a lot of cycles are used in order to guarantee compliance with the
setup and hold times of the SRAM against the we, address and data lines.

\needspace{15\baselineskip}
\begin{verbatim}
==== Chronogram 4.3: 16-bit SRAM writethrough, 32-bit wide =================
                     __    __    __    __    __    __    __    __    __    _
clk                _/  \__/  \__/  \__/  \__/  \__/  \__/  \__/  \__/  \__/ 

cache/ps           ?| (1) | (2) | (3)       | (4) | (5) | (6)       | (7) |?

sram_chip_addr     ?|          218h               |       219h            |?

sram_data_wr       -------|       0000h           |         044Ch         |-
                   _____________             ___________             _______
sram_byte_we_n(0)               \___________/           \___________/
                   _____________             ___________             _______
sram_byte_we_n(1)               \___________/           \___________/

                   _________________________________________________________
sram_oe_n          
============================================================================
\end{verbatim}\\

Signal \emph{cache/ps} is the current state of the cache state machine, and
in this chronogram it takes the following values:

\begin{enumerate}
\item data\_writethrough\_sram\_0a
\item data\_writethrough\_sram\_0b
\item data\_writethrough\_sram\_0c
\item data\_writethrough\_sram\_1a
\item data\_writethrough\_sram\_1b
\item data\_writethrough\_sram\_1c
\end{enumerate}

   

\section{Known Problems}
\label{cache_problems}

    The cache implementation is still provisional and has a number of 
    acknowledged problems:

\begin{enumerate}
\item All parameters hardcoded -- generics are almost ignored.
\item SRAM read state machine does not guarantee internal FPGA $T_{hold}$. 
        In my current target board it works because the FPGA hold times 
        (including an input mux
        in the parent module) are far smaller than the SRAM response times, but
        it would be better to insert an extra cycle after the wait states in
        the sram read state machine.
\item Cache logic mixed with memory controller logic.
\end{enumerate}

