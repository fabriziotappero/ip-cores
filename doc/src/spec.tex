%%%%%%%%%%%%%%%%%%%%%%%%%%%%%%%%%%%%%%%%%%%%%%%%%%%%%%%%%%%%%%%%%%%%%%%%%%%
%%
%% Filename: 	spec.tex
%%
%% Project:	Wishbone Controlled Quad SPI Flash Controller
%%
%% Purpose:	This LaTeX file contains all of the documentation/description
%%		currently provided with this Quad SPI Flash Controller.
%%		It's not nearly as interesting as the PDF file it creates,
%%		so I'd recommend reading that before diving into this file.
%%		You should be able to find the PDF file in the SVN distribution
%%		together with this PDF file and a copy of the GPL-3.0 license
%%		this file is distributed under.
%%		
%%
%% Creator:	Dan Gisselquist
%%		Gisselquist Technology, LLC
%%
%%%%%%%%%%%%%%%%%%%%%%%%%%%%%%%%%%%%%%%%%%%%%%%%%%%%%%%%%%%%%%%%%%%%%%%%%%%
%%
%% Copyright (C) 2015, Gisselquist Technology, LLC
%%
%% This program is free software (firmware): you can redistribute it and/or
%% modify it under the terms of  the GNU General Public License as published
%% by the Free Software Foundation, either version 3 of the License, or (at
%% your option) any later version.
%%
%% This program is distributed in the hope that it will be useful, but WITHOUT
%% ANY WARRANTY; without even the implied warranty of MERCHANTIBILITY or
%% FITNESS FOR A PARTICULAR PURPOSE.  See the GNU General Public License
%% for more details.
%%
%% You should have received a copy of the GNU General Public License along
%% with this program.  (It's in the $(ROOT)/doc directory, run make with no
%% target there if the PDF file isn't present.)  If not, see
%% <http://www.gnu.org/licenses/> for a copy.
%%
%% License:	GPL, v3, as defined and found on www.gnu.org,
%%		http://www.gnu.org/licenses/gpl.html
%%
%%
%%%%%%%%%%%%%%%%%%%%%%%%%%%%%%%%%%%%%%%%%%%%%%%%%%%%%%%%%%%%%%%%%%%%%%%%%%%
\documentclass{gqtekspec}
\project{Quad SPI Flash Controller}
\title{Specification}
\author{Dan Gisselquist, Ph.D.}
\email{dgisselq (at) opencores.org}
\revision{Rev.~0.2}
\begin{document}
\pagestyle{gqtekspecplain}
\titlepage
\begin{license}
Copyright (C) \theyear\today, Gisselquist Technology, LLC

This project is free software (firmware): you can redistribute it and/or
modify it under the terms of  the GNU General Public License as published
by the Free Software Foundation, either version 3 of the License, or (at
your option) any later version.

This program is distributed in the hope that it will be useful, but WITHOUT
ANY WARRANTY; without even the implied warranty of MERCHANTIBILITY or
FITNESS FOR A PARTICULAR PURPOSE.  See the GNU General Public License
for more details.

You should have received a copy of the GNU General Public License along
with this program.  If not, see \hbox{<http://www.gnu.org/licenses/>} for a
copy.
\end{license}
\begin{revisionhistory}
0.2 & 5/26/2015 & Gisselquist & Minor spelling changes\\\hline
0.1 & 5/13/2015 & Gisselquist & First Draft \\\hline
\end{revisionhistory}
% Revision History
% Table of Contents, named Contents
\tableofcontents
\listoffigures
\listoftables
\begin{preface}
The genesis of this project was a desire to communicate with and program an
FPGA board without the need for any proprietary tools.  This includes Xilinx
JTAG cables, or other proprietary loading capabilities such as Digilent's
Adept program.  As a result, all interactions with the board need to take
place using open source tools, and the board must be able to reprogram itself.
\end{preface}

\chapter{Introduction}
\pagenumbering{arabic}
\setcounter{page}{1}

The Quad SPI Flash controller handles all necessary queries and accesses to
and from a SPI Flash device that has been augmented with an additional
two data lines and enabled with a mode allowing all four data lines to
work together in the same direction at the same time.  Since the interface
was derived from a SPI interface, most of the interaction takes place using
normal SPI protocols and only some commands work at the higher four bits
at a time speed.

This particular controller attempts to mask the underlying operation of the
SPI device behind a wishbone interface, to make it so that reads and writes
are as simple as using the wishbone interface.  However, the difference
between erasing (turning bits from '0' to '1') and programming (turning bits
from '1' to '0') breaks this model somewhat.  Therefore, reads from the 
device act like normal wishbone reads, writes program the device and
sort of work with the wishbone, while erase commands require another register
to control.  Please read the Operations chapter for a detailed description
of how to perform these relevant operations.

This controller implements the interface for the Quad SPI flash found on the
Basys-3 board built by Digilent, Inc.  Some portions of the interface may
be specific to the Spansion S25FL032P chip used on this board, and the
100~MHz system clock found on the board, although there is no reason the
controller needs to be limited to this architecture.  It just happens to be
the one I have been designing to and for.  

For a description of how the internals of this core work, feel free to browse
through the Architecture chapter.

The registers that control this core are discussed in the Registers chapter.

As required, you can find a wishbone datasheet in Chapt.~\ref{chap:wishbone}.

The final pertinent information for implementing this core is found in the
I/O Ports chapter, Chapt.~\ref{chap:ioports}.

As always, write me if you have any questions or problems.

\chapter{Architecture}\label{chap:arch}

As built, the core consists of only two components: the wishbone quad SPI
flash controller, {\tt wbqspiflash}, and the lower level quad SPI driver,
{\tt llqspi}.  The controller issues high level read/write commands to the
lower level driver, which actually implements the Quad SPI protocol.

Pictorally, this looks something like Fig.~\ref{fig:arch}.
\begin{figure}\begin{center}\begin{pspicture}(-2in,0)(2in,3.5in)
\rput(0,2.5in){
	\rput(-0.9in,0){\psline{->}(0,1in)(0,0in)}
		\rput[b]{90}(-0.92in,0.5in){\tt i\_wb\_cyc}
	\rput(-0.7in,0){\psline{->}(0,1in)(0,0in)}
		\rput[b]{90}(-0.72in,0.5in){\tt i\_wb\_data\_stb}
	\rput(-0.5in,0){\psline{->}(0,1in)(0,0in)}
		\rput[b]{90}(-0.52in,0.5in){\tt i\_wb\_ctrl\_stb}
	\rput(-0.3in,0){\psline{->}(0,1in)(0,0in)}
		\rput[b]{90}(-0.32in,0.5in){\tt i\_wb\_we}
	\rput(-0.1in,0){\psline{->}(0,1in)(0,0in)}
		\rput[b]{90}(-0.12in,0.5in){\tt i\_wb\_addr}
	\rput( 0.1in,0){\psline{->}(0,1in)(0,0in)}
		\rput[b]{90}( 0.08in,0.5in){\tt i\_wb\_data}
	%
	\rput( 0.5in,0){\psline{<-}(0,1in)(0,0in)}
		\rput[b]{90}( 0.48in,0.5in){\tt o\_wb\_ack}
	\rput( 0.7in,0){\psline{<-}(0,1in)(0,0in)}
		\rput[b]{90}( 0.68in,0.5in){\tt o\_wb\_stall}
	\rput( 0.9in,0){\psline{<-}(0,1in)(0,0in)}
		\rput[b]{90}( 0.88in,0.5in){\tt o\_wb\_data}}
\rput(0,2.0in){%
	\rput(0,0){\psframe(-1.2in,0)(1.2in,0.5in)}
	\rput(0,0.25in){\tt wbqspiflash}}
\rput(0,1.0in){
	\rput(-0.9in,0){\psline{->}(0,1in)(0,0in)}
		\rput[b]{90}(-0.92in,0.5in){\tt spi\_wr}
	\rput(-0.7in,0){\psline{->}(0,1in)(0,0in)}
		\rput[b]{90}(-0.72in,0.5in){\tt spi\_hold}
	\rput(-0.5in,0){\psline{->}(0,1in)(0,0in)}
		\rput[b]{90}(-0.52in,0.5in){\tt spi\_in}
	\rput(-0.3in,0){\psline{->}(0,1in)(0,0in)}
		\rput[b]{90}(-0.32in,0.5in){\tt spi\_len}
	\rput(-0.1in,0){\psline{->}(0,1in)(0,0in)}
		\rput[b]{90}(-0.12in,0.5in){\tt spi\_spd}
	\rput( 0.1in,0){\psline{->}(0,1in)(0,0in)}
		\rput[b]{90}( 0.08in,0.5in){\tt spi\_dir}
	% \rput(-0.9in,0){\psline{->}(0,1in)(0,0in)}
		% \rput[b]{90}(-0.92in,0.5in){\tt i\_wb\_cyc}
	\rput( 0.5in,0){\psline{->}(0,1in)(0,0in)}
		\rput[b]{90}( 0.48in,0.5in){\tt spi\_out}
	\rput( 0.7in,0){\psline{->}(0,1in)(0,0in)}
		\rput[b]{90}( 0.68in,0.5in){\tt spi\_valid}
	\rput( 0.9in,0){\psline{->}(0,1in)(0,0in)}
		\rput[b]{90}( 0.88in,0.5in){\tt spi\_busy}}
\rput(0,0.5in){
	\rput(0,0){\psframe(-1.25in,0)(1.25in,0.5in)}
	\rput(0,0.25in){\tt llqspi}}
	\rput(0,0){\psline{<->}(-0.3in,0.5in)(-0.3in,0)
		\psline{<->}(-0.1in,0.5in)(-0.1in,0)
		\psline{<->}(0.1in,0.5in)(0.1in,0)
		\psline{<->}(0.3in,0.5in)(0.3in,0)}
	\rput[l](0.4in,0.25in){Quad SPI I/O lines}
\end{pspicture}\end{center}
\caption{Architecture Diagram}\label{fig:arch}
\end{figure}
This is also what you will find if you browse through the code.

While it isn't relevant for operating the device, a quick description of these
internal wires may be educational.  The lower level device is commanded by
asserting a {\tt spi\_wr} signal when the device is not busy (i.e. {\tt 
spi\_busy} is low).  The actual command given depends upon the other
signals.  {\tt spi\_len} is a two bit value indicating whether this is an
8 bit (2'b00), 16 bit (2'b01), 24 bit (2'b10), or 32 bit (2'b11) transaction.
The data to be sent out the port is placed into {\tt spi\_in}. 

Further, to support Quad I/O, {\tt spi\_spd} can be set to one to use all four
bits.  In this case, {\tt spi\_dir} must also be set to either 1'b0 for
writing, or 1'b1 to read from the four bits.

When data is valid from the lower level driver, the {\tt spi\_valid} line
will go high and {\tt spi\_out} will contain the data with the most recently
read bits in the lower bits.  Further, when the device is idle, {\tt spi\_busy}
will go low, where it may then read another command.

Sadly, this simple interface as originally designed doesn't work on a 
device where transactions can be longer than 32~bits.  To support these
longer transactions, the lower level driver checks the {\tt spi\_wr} line
before it finishes any transaction.  If the line is high, the lower level
driver will deassert {\tt spi\_busy} for one cycle while reading the command
from the controller on the previous cycle.  Further, the controller can also
assert the {\tt spi\_hold} line which will stop the clock to the device
and force everything to wait for further instructions.

This hold line interface was necessary to deal with a slow wishbone bus that
was writing to the device, but that didn't have it's next data line ready.
Thus, by holding the {\tt i\_wb\_cyc} line high, a write could take many
clocks and the flash would simply wait for it.  (I was commanding the device
via a serial port, so writes could take {\em many} clock cycles for each
word to come through, i.e. 1,500 clocks or so per word and that's at high
speed.)

The upper level component, the controller {\tt wbqspiflash}, is little more
than a glorified state machine that interacts with the wishbone bus.  
From it's idle state, it can handle any command, whether data or control,
and issue appropriate commands to the lower level driver.  From any other
state, it will stall the bus until it comes back to idle--with a few exceptions.
Subsequent data reads, while reading data, will keep the device reading.
Subsequent data writes, while in program mode, will keep filling the devices
buffer before starting the write.  In other respects, the device will just
stall the bus until it comes back to idle.

While they aren't used in this design, the wishbone error and retry signals
would've made a lot of sense here.  Specifically, it should be an error to
read from the device while it is in the middle of an erase or program command.
Instead, this core stalls the bus--trying to do good for everyone.  Perhaps
a later, updated, implementation will make better use of these signals instead
of stalling.  For now, this core just stalls the bus.

Perhaps the best takeaway from this architecture section is that the varying
pieces of complexity have each been separated from each other.  There's a
lower level driver that handles actually toggling the lines to the port,
while the higher level driver maintains the state machine controlling which
commands need to be issued and when.

\chapter{Operation}\label{chap:ops}
This implementation attempts to encapsulate (hide) the details of the chip
from the user, so that the user does not need to know about the various
subcommands going to and from the chip.  The original goal was to make the
chip act like any other read/write memory, however the difference between
erasing and programming the chip made this impossible.  Therefore a separate
register is given to erase any given sector, while reads and writes may proceed
(almost) as normal.

The wishbone bus that this controller works with, however, is a 32--bit
bus.  Address one on the bus addresses a completely different 32--bit word
from address zero or address two.  Bus select lines are not implemented,
all operations are 32--bit.  Further, the device is little--endian, meaning
that the low order byte is the first byte that will be or is stored on the
flash.

\section{High Level}
From a high level perspective, this core provides read/write access to the
device either via the wishbone (read and program), or through a control
register found on the wishbone (the EREG).  Programming the device consists of
first erasing the region of interest.  This will set all the bits to '1' in
that region.  After erasing the region, the region can then be programmed,
setting some of the '1' bits to '0's.  When neither erase nor program
operation is going on, the device may be read.  The section will describe
each of those operations in detail.

To erase a sector of the device, two writes are required to the EREG register.
The first write turns off the write protect bit, whereas the second write
commands the erase itself.  The first write should equal \hbox{0x1000\_0000},
the second should be any address within the sector to be erased together
with setting the high bit of the register or \hbox{0x8000\_0000} plus the
address.  After this second write, the controller will issue a write--enable
command to the device, followed by a sector erase command.  In summary,
\begin{enumerate}
\item Disable write protect by writing \hbox{\tt 0x1000\_0000} to the EREG
	register
\item Command the erase by writing \hbox{\tt 0x8000\_0000} plus the device
	address to the EREG register.  (Remember, this is the {\em word
	address} of interest, not the {\em byte address}.)
\end{enumerate}

While the device is erasing, the controller will idle while checking the
status register over and over again.  Should you wish to read from the EREG
during this time, the high order bit of the EREG register will be set. 
Once the erase is complete, this bit will clear, the interrupt line will
be strobed high, and other operations may take then place on the part.  Any
attempt to perform another operation on the part prior to that time will stall
the bus until the erase is complete.

Once an area has been erased, it may then be programmed.  To program the device,
first disable the write protect by writing a {\tt 0x1000\_0000} to the EREG
register.  After that, you may then write to the area in question whatever
values you wish to program.  One 256~byte (64~bus word) page may be programmed
at a time.  Pages start on even boundaries, such as addresses {\tt 0x040},
{\tt 0x080}, {\tt 0x0100}, etc.  To program a whole page at a time, write the
64~words of the page to the controller without dropping the {\tt i\_wb\_cyc}
line.  Attempts to write more than 64~words will stall the bus, as will
attempts to write more than one page.  Writes of less than a page work as well.
In summary,
\begin{enumerate}
\item Disable the write protect by writing a {\tt 0x1000\_0000} to the EREG
	register.
\item Write the page of interest to the data memory of the device. 

	The first address should start at the beginning of a page (bottom six
	bits zero), and end at the end of the page (bottom six bits one, top
	bits identical).  Writes of less than a page are okay.  Writes crossing
	page boundaries will stall the device.
\end{enumerate}

While the device is programming a page, the controller will idle while
checking the status register as it did during an erase.  During this idle,
both the EREG register and the device status register may be queried.  Once
the status register drops the write in progress line, the top level bit of
the EREG register will be cleared and the interrupt line strobed.  Prior to this
time, any other bus operation will stall the bus until the write completes.

Reads are simple, you just read from the device and the device does everything
you expect.  Reads may be pipelined.  Further, if the device is ever commanded
to read the configuration register, revealing that the quad SPI mode is
enabled, then reads will take place four bits at a time from the bus.
In general, it will take 72 device clocks (at 50~MHz) to read the first word
from memory, and 32 for every pipelined word read thereafter provided that
the reads are in memory order.  Likewise, in quad SPI mode, it will
instead take 28 device clocks to read the first word, and 8 device clocks
to read every word thereafter again provided that the subsequent pipelined
reads are in memory order. 

The Quad SPI device provides for a special mode following a read, where the
next read may start immediately in Quad I/O mode following a 12~clock
setup.  This controller leaves the device in this mode following any initial
read.  Therefore, back to back reads as part of separate bus cycles will only
take 20~clocks to read the first word, and 8~clocks per word thereafter.
Other commands, however, such as erasing, writing, reading from the status,
configuration, or ID registers, will take require a 32~device clock operation
before entering.

\section{Low Level}

At a lower level, this core implements the following Quad SPI commands:
\begin{enumerate}
\item FAST\_READ, when a read is requested and Quad mode has not been enabled.
\item QIOR, or quad I/O high performance read mode.  This is the default read
	command when Quad mode has been enabled, and it leaves the device
	in the Quad I/O High Performance Read mode, ready for a faster second
	read command.
\item RDID, or Read identification
\item WREN, or Write Enable, is issued prior to any erase, program, or
		write register (i.e. configuration or status) command.
	This detail is hidden from the user.
\item RDSR, or read status register, is issued any time the user attempts
	to read from the status register.  Further, following an erase or a
	write command, the device is left reading this register over and over
	again until the write completes.
\item RCR, or read configuration, is issued any time a request is made to
	read from the configuration register.  Following such a read, the
	quad I/O may be enabled for the device, if it is enabled in this
	register.
\item WRR, or write registers, is issued upon any write to the status or
	configuration registers.  To separate the two, the last value read
	from the status register is written to the status register when 
	writing the configuration register.
\item PP, or page program, is issued to program the device in serial mode
	whenever programming is desired and the quad I/O has not been enabled.
\item QPP, or quad page program, is used to program the device whenever
	a write is requested and quad I/O mode has been enabled.
\item SE, or sector erase, is the only type of erase this core supports.
\item CLSR, or Clear Status Register, is issued any time the last status
	register had the bits {\tt P\_ERR} or {\tt E\_ERR} set and the
	write to the status register attempts to clear one of these.  This
	command is then issued following the WRR command.
\end{enumerate}

\chapter{Registers}\label{chap:regs}

This implementation supports four control registers.  These are the EREG
register, the configuration register, the status register, and the device ID,
as shown and listed in Table.~\ref{tbl:reglist}.
\begin{table}[htbp]
\begin{center}
\begin{reglist}
EREG & 0 & 32 & R/W & An overall control register, providing instant status	
	from the device and controlling erase commands.\\\hline
Config & 1 & 8 & R/W & The devices configuration register.\\\hline
Status & 2 & 8 & R/W & The devices status register.\\\hline
ID & 3 & 16 & R & Reads the 16-bit ID from the device.\\\hline
\end{reglist}
\caption{List of Registers}\label{tbl:reglist}
\end{center}\end{table}

\section{EREG Register}
The EREG register was designed to be a replacement for all of the device
registers, leaving all the other registers a part of a lower level access
used only in debugging the device.  This would've been the case, save that
one may need to set bit one of the configuration register to enter high
speed mode.

The bits associated with this register are listed in Tbl.~\ref{tbl:eregbits}.

\begin{table}[htbp]
\begin{center}
\begin{bitlist}
31 & R/W & Write in Progress/Erase.  On a read, this bit will be high if any
	write or erase operation is in progress, zero otherwise.  To erase
	a sector, set this bit to a one.  Otherwise, writes should keep this
	register at zero.\\\hline
30 & R & Dirty bit.  The sector referenced has been written to since it
	was erased.  This bit is meaningless between startup and the first
	erase, but valid afterwards.\\\hline
29 & R & Busy bit.  This bit returns a one any time the lower level Quad
	SPI core is active.  However, to read this register, the lower level
	core must be inactive, so this register should always read zero.
	\\\hline
28 & R/W & Disable write protect.  Set this to a one to disable the write
	protect mode, or to a zero to re--enable write protect on this chip.
	Note that this register is not self--clearing.  Therefore, write
	protection may still be disabled following an erase or a write.
	Clear this manually when you wish to re--enable write protection.
	\\\hline
27 & R & Returns a one if the device is in high speed (4-bit I/O) mode.
	To set the device into high speed mode, set bit~1 of the configuration
	register.\\\hline
20--26 & R & Always return zero.\\\hline
14--19 & R/W & The sector address bits of the last sector erased.  If the
	erase line bit is set while writing this register, these bits
	will be set as well with the sector being erased.\\\hline
0--13 & R & Always return zero.\\\hline
\end{bitlist}
\caption{EREG bit definitions}\label{tbl:eregbits}
\end{center}\end{table}

In general, only three bits and an address are of interest here. 

The first bit of interest is bit 27, which will tell you if you are in Quad--I/O
mode.  The device will automatically start up in SPI serial mode.  Upon 
reading the configuration register, it will transition to Quad--I/O mode if
the QUAD bit is set.  Likewise, if the bit is written to the configuration
register it will transition to Quad--I/O mode.

While this may seem kind of strange, I have found this setup useful.  It allows
me to debug commands that might work in serial mode but not quad I/O mode,
and it allows me to explicitly switch to Quad I/O mode.  Further, writes to the
configuration register are non--volatile and in some cases permanent. 
Therefore, it doesn't make sense that a controller should perform such a write
without first being told to do so.  Therefore, this bit is set upon
noticing that the QUAD bit is set in the configuration register.

The second bit of interest is the write protect disable bit.  Write a '1'
to this bit before any erase or program operation, and a '0' to this bit
otherwise.  This allows you to make sure that accidental bus writes to the
wrong address won't reprogram your flash (which they would do otherwise).

The final bit of interest is the write in progress slash erase bit.  On read,
this bit mirrors the WIP bit in the status register.  It will be a one during
any ongoing erase or programming operation, and clear otherwise.  Further,
to erase a sector, disable the write protect and then set this bit to a one
while simultaneously writing the sector of interest to the device.

The last item of interest in this register is the sector address of interest.
This was placed in bits 14--19 so that any address within the sector
would work.  Thus, to erase a sector, write the sector address, together with
an erase bit, to this register.

\section{Config Register}

The Quad Flash device also has a non--volatile configuration register, as
shown in Tbl.~\ref{tbl:confbits}.  Writes to this register are program events, 
which will stall subsequent bus operations until the write in progress bit
of either the status or EREG registers clears.  Note that some bits, once
written, cannot be cleared such as the BPNV bit.

Writes to this register are not truly independent of the status register,
as the Write Registers (WRR) command writes the status register before the
configuration register.  Therefore, the core implements this by writing the
status register with the last value that was read by the core, or zero
if the status register has yet to be read by the core.  Following the
status register write, the new value for the configuration register is
written.
\begin{table}[htbp]\begin{center}
\begin{bitlist}
8--31 & R & Always return zero.\\\hline
6--7 & R & Not used.\\\hline
5 & R/W & TBPROT. Configures the start of block protection.  See device
	documentation for more information.  (Default 0)\\\hline
4 & R/W & Do not use.  (Default 0)\\\hline
3 & R/W & BPNV, configures BP2--0 bits in the status register.  If this bit
	is set to 1, these bits are volatile, if set to '0' (default) the
	bits are non--volatile.  {\em Note that once this bit has been set,
	it cannot be cleared!}\\\hline
2 & R/W & TBPARM.  Configures the parameter sector location.  See device
	documentation for more detailed information.  (Default 0)\\\hline
1 & R/W & QUAD.  Set to '1' to place the device into Quad I/O (4--bit) mode,
	'0' to leave in dual or serial I/O mode.  (This core does not support
	dual I/O mode.)  (Most programmers will set this to '1'.)\\\hline
0 & R/W & FREEZE.  Set to '1' to lock bits BP2--0 in the status register, zero
	otherwise.  (Default 0).\\\hline
	\\\hline
\end{bitlist}
\caption{Configuration bit definitions}\label{tbl:confbits}
\end{center}\end{table}

Further information on this register is available in the device data sheet.

\section{Status Register}
The definitions of the bits in the status register are shown in
Tbl.~\ref{tbl:statbits}.  For operating this core, only the write in progress
bit is relevant.  All other bits should be set to zero.

\begin{table}[htbp]
\begin{center}
\begin{bitlist}
8--31 & R & Always return zero.\\\hline
7 & R/W & Status register write disable.  This setting is irrelevant in the
	current core configuration, since the W\#/ACC line is always kept
	high.\\\hline
6 & R/W & P\_ERR.  The device will set this to a one if a programming error
	has occurred.  Writes with either P\_ERR or E\_ERR cleared will
	clear this bit.\\\hline
5 & R/W & E\_ERR.  The device will set this to a one if an erase error has
	occurred, zero otherwise.  Writes clearing either P\_ERR or E\_ERR
	will clear this bit.
	\\\hline
2--4 & R/W & Block protect bits.  This core assumes these bits are zero.
	See device documentation for other possible settings.\\\hline
1 & R & Write Enable Latch.  This bit is handled internally by the core,
	being set before any program or erase operation and cleared by
	the operation itself.  Therefore, reads should always read this
	line as low.\\\hline
0 & R & Write in Progress.  This bit, when one, indicates that an erase or
	program operation is in progress.  It will be cleared upon completion.
	\\\hline
\end{bitlist}
\caption{Status bit definitions}\label{tbl:statbits}
\end{center}\end{table}

\section{Device ID}

Reading from the Device ID register causes the core controller to issue
a RDID {\tt 0x9f} command.  The bytes returned are first the manufacture
ID of the part ({\tt 0x01} for this part), followed by the device ID
({\tt 0x0215} for this part), followed by the number of extended bytes that
may be read ({\tt 0x4D} for this part).  This controller provides no means
of reading these extended bytes.  (See Tab.~\ref{tbl:idbits})

\begin{table}[htbp]\begin{center}
\begin{bitlist}
0--31 & R & Always reads {\tt 0x0102154d}.\\\hline
\end{bitlist}
\caption{Read ID bit definitions}\label{tbl:idbits}
\end{center}\end{table}

\chapter{Wishbone Datasheet}\label{chap:wishbone}
Tbl.~\ref{tbl:wishbone} is required by the wishbone specification, and so 
it is included here.
\begin{table}[htbp]
\begin{center}
\begin{wishboneds}
Revision level of wishbone & WB B4 spec \\\hline
Type of interface & Slave, (Block) Read/Write \\\hline
Port size & 32--bit \\\hline
Port granularity & 32--bit \\\hline
Maximum Operand Size & 32--bit \\\hline
Data transfer ordering & Little Endian \\\hline
Clock constraints & Must be 100~MHz or slower \\\hline
Signal Names & \begin{tabular}{ll}
		Signal Name & Wishbone Equivalent \\\hline
		{\tt i\_clk\_100mhz} & {\tt CLK\_I} \\
		{\tt i\_wb\_cyc} & {\tt CYC\_I} \\
		{\tt i\_wb\_ctrl\_stb} & {\tt STB\_I} \\
		{\tt i\_wb\_data\_stb} & {\tt STB\_I} \\
		{\tt i\_wb\_we} & {\tt WE\_I} \\
		{\tt i\_wb\_addr} & {\tt ADR\_I} \\
		{\tt i\_wb\_data} & {\tt DAT\_I} \\
		{\tt o\_wb\_ack} & {\tt ACK\_O} \\
		{\tt o\_wb\_stall} & {\tt STALL\_O} \\
		{\tt o\_wb\_data} & {\tt DAT\_O}
		\end{tabular}\\\hline
\end{wishboneds}
\caption{Wishbone Datasheet for the Quad SPI Flash controller}\label{tbl:wishbone}
\end{center}\end{table}

\chapter{Clocks}\label{chap:clocks}

This core is based upon the Basys--3 design.  The Basys--3 development board
contains one external 100~MHz clock.  This clock is divided by two to create
the 50~MHz clock used to drive the device.   According to the data sheet,
it should be possible to run this core at up to 160~MHz, however I have not
tested it at such speeds.  See Table.~\ref{tbl:clocks}.
\begin{table}[htbp]
\begin{center}
\begin{clocklist}
i\_clk\_100mhz & External & 160 & & System clock.\\\hline
\end{clocklist}
\caption{List of Clocks}\label{tbl:clocks}
\end{center}\end{table}

\chapter{I/O Ports}\label{chap:ioports}
There are two interfaces that this device supports: a wishbone interface, and
the interface to the Quad--SPI flash itself.  Both of these have their own
section in the I/O port list.  For the purpose of this table, the wishbone
interface is listed in Tbl.~\ref{tbl:iowishbone}, and the Quad SPI flash
interface is listed in Tbl.~\ref{tbl:ioqspi}.  The two lines that don't really
fit this classification are found in Tbl.~\ref{tbl:ioother}.
\begin{table}[htbp]
\begin{center}
\begin{portlist}
i\_wb\_cyc & 1 & Input & Wishbone bus cycle wire.\\\hline
i\_wb\_data\_stb & 1 & Input & Wishbone strobe, when the access is to the data 
		memory.\\\hline
i\_wb\_ctrl\_stb & 1 & Input & Wishbone strobe, for when the access is to 
	one of control registers.\\\hline
i\_wb\_we & 1 & Input & Wishbone write enable, indicating a write interaction
		to the bus.\\\hline
i\_wb\_addr & 19 & Input & Wishbone address.  When accessing control registers,
		only the bottom two bits are relevant all other bits are 
		ignored.\\\hline
i\_wb\_data & 32 & Input & Wishbone bus data register.\\\hline
o\_wb\_ack & 1 & Output & Return value acknowledging a wishbone write, or
		signifying valid data in the case of a wishbone read request.
		\\\hline
o\_wb\_stall & 1 & Output & Indicates the device is not yet ready for another
		wishbone access, effectively stalling the bus.\\\hline
o\_wb\_data & 32 & Output & Wishbone data bus, returning data values read
		from the interface.\\\hline
\end{portlist}
\caption{Wishbone I/O Ports}\label{tbl:iowishbone}
\end{center}\end{table}

While this core is wishbone compatible, there was one necessary change to
the wishbone interface to make this possible.  That was the split of the
strobe line into two separate lines.  The first strobe line, the data strobe,
is used when the access is to data memory--such as a read or write (program)
access.  The second strobe line, the control strobe, is for reads and writes
to one of the four control registers.  By splitting these strobe lines,
the wishbone interconnect designer may place the control registers in a
separate location of wishbone address space from the flash memory.  It is
an error for both strobe lines to be on at the same time.

With respect to the Quad SPI interface itself, one piece of glue logic
is necessary to tie the Quad SPI flash I/O to the in/out port at the top
level of the device.  Specifically, these two lines must be added somewhere:
\begin{tabbing}
assign {\tt io\_qspi\_dat} = \= (\~{\tt qspi\_mod[1]})?(\{2'b11,1'bz,{\tt qspi\_dat[0]}\}) \hbox{\em // Serial mode} \\
	\> :(({\tt qspi\_bmod[0]})?(4'bzzzz):({\tt qspi\_dat[3:0]}));
		\hbox{\em // Quad mode}
\end{tabbing}
These provide the transition between the input and output ports used by this 
core, and the bi--directional inout ports used by the actual part.  Further,
because the two additional lines are defined to be ones during serial I/O
mode, the hold and write protect lines are effectively eliminated in this
design in favor of faster speed I/O (i.e., Quad I/O).

\begin{table}[htbp]
\begin{center}
\begin{portlist}
o\_qspi\_sck & 1 & Output & Serial clock output to the device.  This pin
		will be either inactive, or it will toggle at 50~MHz.\\\hline
o\_qpsi\_cs\_n & 1 & Output & Chip enable, active low.  This will be
		set low at the beginning of any interaction with the chip,
		and will be held low throughout the interaction.\\\hline
o\_qspi\_mod & 2 & Output & Two mode lines for the top level to control
	how the output data lines interact with the device.  See the text
	for how to use these lines.\\\hline
o\_qspi\_dat & 4 & Output & Four output lines, the least of which is the
	old SPI MOSI line.  When selected by the o\_qspi\_mod, this output
	becomes the command for all 4 QSPI I/O lines.\\\hline
i\_qspi\_dat & 4 & Input & The four input lines from the device, of which
	line one, {\tt i\_qspi\_dat[1]}, is the old MISO line.\\\hline
\end{portlist}
\caption{List of Quad--SPI Flash I/O ports}\label{tbl:ioqspi}
\end{center}\end{table}

Finally, the clock line is not specific to the wishbone bus, and the interrupt
line is not specific to any of the above.  These have been separated out here.
\begin{table}[htbp]
\begin{center}
\begin{portlist}
i\_clk\_100mhz & 1 & Input & The 100~MHz clock driving all interactions.\\\hline
o\_interrupt & 1 & Output & An strobed interrupt line indicating the end of
	any erase or write transaction.  This line will be high for exactly
	one clock cycle, indicating that the core is again available for
	commanding.\\\hline
\end{portlist}
\caption{Other I/O Ports}\label{tbl:ioother}
\end{center}\end{table}
% Appendices
% Index
\end{document}


