%%%%%%%%%%%%%%%%%%%%%%%%%%%%%%%%%%%%%%%%%%%%%%%%%%%%%%%%%%%%%%%%%%%%%%%%%%%%%%%%
%%            _   _            __   ____                                      %%
%%           / / | |          / _| |  __|                                     %%
%%           | |_| |  _   _  / /   | |_                                       %%
%%           |  _  | | | | | | |   |  _|                                      %%
%%           | | | | | |_| | \ \_  | |__                                      %%
%%           |_| |_| \_____|  \__| |____| microLab                            %%
%%                                                                            %%
%%           Bern University of Applied Sciences (BFH)                        %%
%%           Quellgasse 21                                                    %%
%%           Room HG 4.33                                                     %%
%%           2501 Biel/Bienne                                                 %%
%%           Switzerland                                                      %%
%%                                                                            %%
%%           http://www.microlab.ch                                           %%
%%%%%%%%%%%%%%%%%%%%%%%%%%%%%%%%%%%%%%%%%%%%%%%%%%%%%%%%%%%%%%%%%%%%%%%%%%%%%%%%
\chapter{Bus transactions}
\label{appen:bus}
This appendix describes the timing diagram of valid bus transactions and
transactions containing errors. Although the timing diagrams are shown for a
three datum burst transaction, it also reflects the single datum transaction.
%-----------------------------------------------------------------------------
\section{Write transactions}
In the write transaction the data flows from the user FPGA towards the {\sc
GECKO4com}. The timing diagram of a correct write transaction is shown in
Figure~\ref{fig:write correct}.
A write transaction is initiated by activating the $\overline{\textbf{start
trans}}$ signal and sending the \emph{Transmission Control Word (TCW)} (see
Chapter~\ref{sec:bus prot} and Figure~\ref{fig:TCW}) over the \textbf{data
cntrl} lines.\\
\textit{Important: All signals are active for one clock period of the
\textbf{bus clock}.\important}
\begin{figure}[hb]
\centering%
\includegraphics[width=\columnwidth]{figs/write_transaction_no_bus_error}
\caption{A correct write transaction that causes no bus error. Here a 3-short
burst transaction is shown. The blue lines represent tri-stated FPGA pins.}
\label{fig:write correct}
\end{figure}
 \\
After the initiation of the write transaction the user FPGA has to wait for the
{\sc GECKO4com} to activate the $\overline{\textbf{start send}}$ signal
(dependency \ding{'312}). Sending data before this dependency will result in
unpredictable results. After the reception of the $\overline{\textbf{start
send}}$ signal the user FPGA may start transmitting the data payload(shown by
\ding{'313}). The data payload may be a continues stream or chunk-ed into parts.
For each datum the user FPGA has to put the datum on the \textbf{data cntrl}
lines and activate both the $\overline{\textbf{valid lo}}$ and
$\overline{\textbf{valid hi}}$ lines to indicate valid data. After the sending
of the last datum, the user FPGA has to end the transaction by activating the
$\overline{\textbf{end trans}}$ signal as shown at \ding{'314}. The
$\overline{\textbf{end trans}}$ signal may be activated after or in parallel
with the sending of the last datum of the payload.\\
\textit{Important: the {\sc GECKO4com} does not check whether or not the user
FPGA sends the correct number of data. It assumes that the user FPGA does send
the number of data as announced in the Transmission Control Word (TCW).\important}
%-----------------------------------------------------------------------------
\section{Write transaction aborts}
Write transactions can be aborted under the conditions described in
Chapter~\ref{sec:mem map}. An aborted transaction is indicated by the {\sc
GECKO4com} by the activation of the $\overline{\textbf{error}}$ line. This
condition can occur anywhere during the transaction. Figure~\ref{fig:write
error 1} and Figure~\ref{fig:write error 2} show two examples of aborted write
transactions.
\begin{figure}[pt]
\centering%
\includegraphics[width=\columnwidth]{figs/write_transaction_bus_error_1}
\caption{An aborted write transaction before the reception of the
$\overline{\textbf{start trans}}$ signal.}
\label{fig:write error 1}
\end{figure}
\begin{figure}[pb]
\centering%
\includegraphics[width=\columnwidth]{figs/write_transaction_bus_error_2}
\caption{An aborted write transaction after the reception of the
$\overline{\textbf{start trans}}$ signal.}
\label{fig:write error 2}
\end{figure}
After the reception of the activated $\overline{\textbf{error}}$ signal the user
FPGA is required to activate the $\overline{\textbf{end trans}}$ signal to end
the transaction (as shown in dependency \ding{'315}).\\
\textit{Important: Failing to adhere to dependency \ding{'315} may leave the bus
in an undefined state.\important}
%-----------------------------------------------------------------------------
\section{Read transactions}
Simular to the write transaction a read transaction is initiated  by activating the $\overline{\textbf{start
trans}}$ signal and sending the \emph{Transmission Control Word (TCW)} (see
Chapter~\ref{sec:bus prot} and Figure~\ref{fig:TCW}) over the \textbf{data
cntrl} lines. As during the read transaction the data has to flow from the {\sc
GECKO4com} to the user FPGA the controlling of the bi-directional signals, shown
in  Table~\ref{tab:gecko4 bus signals}, need special attention.
Figure~\ref{fig:read correct} depicts a none-aborted read transaction for a
burst of three. 
\begin{figure}[t]
\centering%
\includegraphics[width=\columnwidth]{figs/read_transaction_no_bus_error}
\caption{A correct read transaction that causes no bus error. Here a 3-short
burst transaction is shown. The blue lines represent tri-stated FPGA pins.}
\label{fig:read correct}
\end{figure}
One cycle after the initialization of a read transaction (\ding{206}) the user
FPGA has to tristate all bi-directional signals. The user FPGA has to keep the
bi-directional signals in tristate up to one cycle after receiving an activated
$\overline{\textbf{end trans}}$ signal (\ding{208}).\\
\textit{Important: Failing to adhere to dependency \ding{206} and dependency \ding{208} may
destroy the IOB buffers of either or both the user FPGA and the {\sc GECKO4com}
FPGA!\important}\\
Two cycles after the reception of a read transaction (\ding{206}) the {\sc GECKO4com} will
start driving the bi-directional signals. The {\sc GECKO4com} will send the
requested number of shorts in one continues burst (\ding{207}) over the \textbf{data cntrl}
lines and activates the $\overline{\textbf{valid low}}$ and
$\overline{\textbf{valid low}}$ as described in Chapter~\ref{sec:bus prot}. During the transmission of the last datum of the read transaction
the {\sc GECKO4com} ends the transmission by activation of the
$\overline{\textbf{end trans}}$ signal. The \textbf{bus clock} cycle after the
activation of the $\overline{\textbf{end trans}}$ signal the {\sc GECKO4com}
puts all the bi-directional signals in three-state.\newpage
\textit{Important: The user FPGA has to make sure it has enough buffer capacity
to store the requested amount of data, as there is no way to interrupt the data
flow coming from the {\sc GECKO4com}.\important}
%-----------------------------------------------------------------------------
\section{Read transactions abort}
Read transactions can be aborted under the conditions described in
Chapter~\ref{sec:mem map}. An aborted transaction is indicated by the {\sc
GECKO4com} by the activation of the $\overline{\textbf{error}}$ line together
with the activation of the $\overline{\textbf{end trans}}$ signal (\ding{209}). 
Figure~\ref{fig:read error} shows an aborted read transaction.
\begin{figure}[t]
\centering%
\includegraphics[width=\columnwidth]{figs/read_transaction_bus_error}
\caption{An aborted read transaction. The blue lines represent tri-stated FPGA pins.}
\label{fig:read error}
\end{figure}
