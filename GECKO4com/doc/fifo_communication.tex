%%%%%%%%%%%%%%%%%%%%%%%%%%%%%%%%%%%%%%%%%%%%%%%%%%%%%%%%%%%%%%%%%%%%%%%%%%%%%%%%
%%            _   _            __   ____                                      %%
%%           / / | |          / _| |  __|                                     %%
%%           | |_| |  _   _  / /   | |_                                       %%
%%           |  _  | | | | | | |   |  _|                                      %%
%%           | | | | | |_| | \ \_  | |__                                      %%
%%           |_| |_| \_____|  \__| |____| microLab                            %%
%%                                                                            %%
%%           Bern University of Applied Sciences (BFH)                        %%
%%           Quellgasse 21                                                    %%
%%           Room HG 4.33                                                     %%
%%           2501 Biel/Bienne                                                 %%
%%           Switzerland                                                      %%
%%                                                                            %%
%%           http://www.microlab.ch                                           %%
%%%%%%%%%%%%%%%%%%%%%%%%%%%%%%%%%%%%%%%%%%%%%%%%%%%%%%%%%%%%%%%%%%%%%%%%%%%%%%%%
\chapter{User FPGA communication}
\label{chap:fpga}
The {\sc GECKO4com} provides two FIFOs of each four kilo bytes to stream data
from the PC to the user FPGA and vice versa. These FIFOs can operate in the
IEEE488.x mode or in transparent mode. 
This chapter will first introduce how to steam data through
these FIFOs. After that it will describe the IEEE488.x mode. It will end with
the description of the transparent mode.
%-------------------------------------------------------------------------------
\section{FIFO communication}
\label{sec:fifo com}
The {\sc GECKO4com} uses the {\sc usbtmc} protocol to stream data over an USB
connection. This {\sc usbtmc} protocol has three important aspects:
\begin{itemize}
\item \textbf{Packet format}. Each {\sc usbtmc} packet consists of a header and
data payload. In all cases the {\sc GECKO4com} handles the header
generation/interpretation. The user FPGA only sees/provides the data payload.
\item \textbf{Payload format}. The data payload as provided in an {\sc usbtmc}
packet is byte based, e.g. can be any number of bytes (minimal one).
\item \textbf{Message size}. Messages that are send/received can be divided into
multiple {\sc usbtmc} packets.
\end{itemize}
To be able to be as versatile as possible, the {\sc GECKO4com} imposes that the
messages send to/received from the user FPGA {\sc must} be word aligned. This
means that the \textbf{message size} {\sc must} always be a multiple of four bytes.
Failing to do so may result in unpredictable results, or even hang the {\sc
GECKO4com}.

As the {\sc usbtmc} protocol does not provide the means to send the size of a 
message with the first {\sc usbtmc} packet, special care has to be taken when
reading messages coming from the PC. Messages coming from the PC are stored in
the Rx-FIFO. The Rx-FIFO can be accessed by the memory-mapped registers as
described in Chapter~\ref{sec:mem map}. For the user FPGA to be able to
determine the end of a message, the {\sc GECKO4com} marks the last short (2
bytes) of the message with a special valid combination (see Chapter~\ref{sec:bus
prot}).\\
\textit{Important: An empty Rx FIFO does not mean that a message has ended, only
the special valid combination indicates the end of a send message.\important}

Similar to the reception of messages, also the transmission of messages can be
split-up in multiple {\sc usbtmc} packets. For the {\sc GECKO4com} to know the
size of a transmitted message, the user FPGA must write first the size of the
message (one or two short(s) [2 bytes]) prior to putting the messages bytes in
the Tx-FIFO. The size register and the Tx-FIFO can be accessed by the
memory-mapped registers as described in Chapter~\ref{sec:mem map}.\\
\textit{Important: The user FPGA {\sc must} send as many bytes as it indicated
in the size register. If the Tx-FIFO is full, the user FPGA {\sc must} wait
until it can send again data to the Tx-FIFO.\important}
%-------------------------------------------------------------------------------
\section{IEEE488.x mode}
The \emph{IEEE488.x} mode is the default mode of the {\sc GECKO4com}. In this
mode it provides IEEE488.x compliancy and the SCPI command interpretation of the
commands described in Chapter~\ref{chap:commands}. In this mode messages to and
from the user FPGA can {\sc only} be send by using the \verb+FIFO+ and
\verb+FIFO?+ command set; hence the PC forms the master and the user FPGA has
the role of slave.\\
 \\
\textbf{Sending messages from the PC to the user FPGA}\\
Before explaining the protocol involved in sending messages it is emphased that,
 as described in the previous section, the size of a message should be at least
4 bytes and word aligned (meaning an integer multiple of 4 bytes). Although
the {\sc GECKO4com} will fill up the message with up to 3 dummy bytes in case of
word-mis-alignment, no guarantees are given on proper operation.

The sending of a message is initiated by invoking the \verb*+FIFO <message>+
command at the PC side. Between the command and the message only a single space
(\verb*+ +) is allowed; all other spaces are interpreted as being part of the
message. The message (\verb+<message>+) itself may be any stream of bytes, there
are no restrictions as all bytes of the message will be send 1-to-1 to the
Rx-FIFO of the {\sc GECKO4com}. At the moment the {\sc GECKO4com} receives the
\verb+FIFO+ command it will activate the \textbf{Data Available Interrupt (DAI)} 
(see Chapter~\ref{sec: irqs}) for one \textbf{bus clock} cycle. Directly after
the generation of the DAI the {\sc GECKO4com} will start copying the message to
the Rx-FIFO.

The user FPGA {\sc must} after the reception of the DAI start reading the
message data from Rx-FIFO within one second. Failing to do so may result in a
communication time-out in the {\sc usbtmc} communication channel. The user FPGA
must read data from the Rx-FIFO up to the short that is marked as last
short of the message (see Chapter~\ref{sec:bus prot}). To avoid time-outs in the
{\sc usbtmc} communication channel the user FPGA has to read at least 512 bytes
each second.\\
 \\
\textbf{Receiving messages from the user FPGA}\\
To receive a message from the user FPGA the PC has to initiate the communication
by sending the \verb+FIFO?+ command to the {\sc GECKO4com}. After the reception
of this command the {\sc GECKO4com} will activate the \textbf{Data Request
Interrupt (DRI)} (see Chapter~\ref{sec: irqs}) for one \textbf{bus clock} cycle.
After the generation of the DRI the {\sc GECKO4com} will wait for the message
from the user FPGA and transfer it to the PC.\\
The user FPGA must after the reception of the DAI send the message size
and some message bytes within one second. Failing to do so may result in a
communication time-out in the {\sc usbtmc} communication channel. The user FPGA
starts the sending of the message by first writing its size to the size register
at address \verb+0x01+ (see Chapter~\ref{sec:mem map}). The size is written by
first writing the low short (2 bytes) to address \verb+0x01+ followed by the
writing of the high short to address \verb+0x01+. Example: The message size is
\verb+0x000A1234+ bytes. Sending this message would require writing (1) the
value \verb+0x1234+ to address \verb+0x01+, followed by (2) writing the value
\verb+0x000A+ to address \verb+0x01+, followed by (3) writing all the messages
bytes to address \verb+0x00+. The user FPGA must make sure that if the Tx-FIFO
is not full it fills it with at least 512 bytes each second; failing in doing so
may trigger a time-out in the {\sc usbtmc} communication channel.
%-------------------------------------------------------------------------------
\section{Transparent mode}
\label{sec:trans mode}
The {\sc GECKO4com} can be put into transparent mode by invoking the
\verb+TRANS+ command (see Chapter~\ref{chap:commands}). Once put in transparent
mode the {\sc GECKO4com} can only be put back into \emph{IEEE488.x} mode by power
cycling the board, resetting the board, or invoking an \verb+INITIATE_CLEAR+
{\sc usbtmc} message.\\
\textit{Important: The {\sc GECKO4com} will only switch to transparent mode if
the user FPGA is configured. If the user FPGA is not configured an execution
error is generated and the {\sc GECKO4com} stays in IEEE488.x mode.\important}\\

When in transparent mode the {\sc GECKO4com} will disable all command
interpretations; all commands, as listed in Chapter~\ref{chap:commands}, are
interpreted as part of a message and are not executed. In the transparent mode
all messages send from the PC are directly copied to the Rx-FIFO. Each new
message will trigger a \textbf{Data Available Interrupt (DAI)}. The sending of
messages from the user FPGA to the PC can be done by writing the message size
followed by the message bytes themselves in the Tx-FIFO.\\
\textit{Note: In the transparent mode the Data Request Interrupt (DRI) has no
function.\note}\\

This mode is useful when the user wants to implements a new SCPI command
interpreter (see also Chapter~\ref{sec: SBR} concerning the Status Byte
Register) or an own protocol on top of the {\sc usbtmc} protocol.
